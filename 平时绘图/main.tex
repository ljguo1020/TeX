\documentclass{standalone}
\usepackage{ctex}
\usepackage{tikz}
\definecolor{backn}{RGB}{127,0,255}
\definecolor{frame}{RGB}{153,0,153}
\colorlet{back}{backn!50!frame}
\begin{document}
    \begin{tikzpicture}
        \tikzset{every node/.style={align=center,fill=back,line width=1pt,text=white,font=\bfseries,minimum height=4em}}
        \tikzset{arrow/.style={-latex,line width=1pt}}
        \node[minimum width=10em] (a) at (0,0) {绪论\\[-2pt](第一章)};
        \node[minimum width=30em] (b) at ([yshift=-2.7cm]a) {战术导弹内外弹道一体化仿真建模\\[-2pt](第二章)};
        \node[minimum width=18em] (c) at ([yshift=-2.7cm,xshift=-4.3cm]b){克里金代理模型\\[-2pt](第三章)};
        \node[minimum width=18em] (d) at ([yshift=-2.7cm,xshift=4.3cm]b){一种改进的正弦余弦智能搜索算法\\[-2pt](第四章)};
        \node[minimum width=30em] (e) at ([yshift=-8.1cm]a) {地地战术导弹内外弹道一体化总体优化设计\\[-2pt](第五章)};
        \node[minimum width=10em] (f) at ([yshift=-2.7cm]e) {结论与展望\\[-2pt](第六章)};
        \draw[arrow] (a.south)--(b.north);
        \draw[arrow] ([xshift=-4.3cm]b.south)--(c.north);
        \draw[arrow] ([xshift=4.3cm]b.south)--(d.north);
        \draw[arrow] (c.south)--+(0,-1.18);
        \draw[arrow] (d.south)--+(0,-1.18);
        \draw[arrow] (e.south)--(f.north);
    \end{tikzpicture}
\end{document}