\documentclass{ctexart}

\setCJKmainfont{TW-MOE-Li}%设置字体,由于版权问题,请自行设置

\usepackage{xcolor}
%定义主颜色
\definecolor{main}{RGB}{218 ,165 ,32}

\usepackage{tikz} 
% 使用tikz的shadow库,做出旗子的阴影效果
\usetikzlibrary{shadows}
%去掉页码
\pagestyle{empty}

% 定义一些命令,具体效果如下
\newcommand{\key}[1]{
       \draw[line width=0.7pt,main!60!black] ([shift={(#1)}]O.north west)++(0.05,0.25)|- ++(0.2,-0.2);
       \draw[line width=0.7pt,main!60!black] ([shift={(#1)}]O.north west)++(-0.05,0.25)|- ++(-0.2,-0.2);
       \draw[line width=0.7pt,main!60!black] ([shift={(#1)}]O.north west)++(-0.05,-0.27)|- ++(-0.2,0.2);
       \draw[line width=0.7pt,main!60!black] ([shift={(#1)}]O.north west)++(0.05,-0.27)|- ++(0.2,0.2);
       }

\newcommand{\keyleft}[1]{
       \draw[line width=0.7pt,main!60!black] ([shift={(#1)}]O.north west)++(-0.05,0.25)|- ++(-0.2,-0.2);
       \draw[line width=0.7pt,main!60!black] ([shift={(#1)}]O.north west)++(-0.05,-0.27)|- ++(-0.2,0.2); 
       }

\newcommand{\keyright}[1]{
       \draw[line width=0.7pt,main!60!black] ([shift={(#1)}]O.north west)++(0.05,0.25)|- ++(0.2,-0.2);
       \draw[line width=0.7pt,main!60!black] ([shift={(#1)}]O.north west)++(0.05,-0.27)|- ++(0.2,0.2);
       }

%  定义旗子,为了更具立体感,使用了shadow并填充渐变色

\newcommand{\qizi}[2]{
    \path[drop shadow={fill=black,opacity=0.5,shadow xshift=5pt,shadow yshift=-5pt},outer color=main!85!black,inner color=white] ([shift={(#1)}]O.north west)circle(0.6cm);
    \path[fill=main,draw=main!60!black] ([shift={(#1)}]O.north west)circle(0.5cm);
    \node[font=\huge\bfseries] at ([shift={(#1)}]O.north west) {#2};
}

\begin{document}
\pagecolor{main}
    \begin{tikzpicture}[remember picture,overlay]

       %命名并填充整个节点
       \node[fill=main,minimum width=10cm,minimum height=11cm,inner sep=0pt] (O)at (current page.center) {}; 
      %画最外层矩形框
       \draw[line width=3pt,main!50!black] ([shift={(0.8,0.8)}]O.south west) rectangle ([shift={(-0.8,-0.8)}]O.north east);
      %画内层的矩形框
       \draw[line width=1pt,main!60!black]([shift={(1,1)}]O.south west) rectangle ([shift={(-1,-1)}]O.north east);
       %小方格
       \foreach \x in{2,...,9}
       \draw[line width=1pt,main!60!black] ([shift={(1,\x)}]O.south west)--++(8,0);

       \foreach \y in{2,...,8}
       \draw[line width=1pt,main!60!black] ([shift={(\y,1)}]O.south west)--++(0,4) ([shift={(\y,6)}]O.south west)--++(0,4);
        %标记顶部的数字
       \foreach \z in{1,...,9}
       \node[scale=1.2,above=4pt] at ([shift={(\z,-1)}]O.north west) {\z};
        %绘制四条对角线
       \draw[line width=1pt,main!60!black]([shift={(4,-1)}]O.north west)--++(2,-2);

       \draw[line width=1pt,main!60!black]([shift={(6,-1)}]O.north west)--++(-2,-2);

       \draw[line width=1pt,main!60!black]([shift={(4,1)}]O.south west)--++(2,2);

       \draw[line width=1pt,main!60!black]([shift={(6,1)}]O.south west)--++(-2,2);
        
   
        %一些文字的填充
       \node[font=\huge\bfseries,opacity=0.6,scale=0.7] at ([xshift=-2cm,yshift=-2pt]O.center) {楚河};

       \node[font=\huge\bfseries,opacity=0.6,scale=0.7] at ([xshift=2cm,yshift=-2pt]O.center) {漢界};

       \node[font=\huge\bfseries,scale=0.8] at ([yshift=1cm]O.center) {真羨慕那些有對象的人};

       \node[font=\huge\bfseries,scale=0.8] at ([yshift=-1cm]O.center) {我除了帥一無所有};
        % 旗子
        \qizi{3,-1}{象}
        \qizi{5,-1}{將}
        \qizi{5,-3}{象}
        \qizi{5,-10}{帥}
        
        \key{2,-3}
        \key{8,-3}
        \key{2,-8}
        \key{8,-8}
        \key{3,-4}
        \key{5,-4}
        \key{7,-4}
        \key{3,-7}
        \key{5,-7}
        \key{7,-7}
        \keyleft{9,-4}
        \keyleft{9,-7}
        \keyright{1,-4}
        \keyright{1,-7}
        
       
     
    \end{tikzpicture}
\end{document}