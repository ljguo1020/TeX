\documentclass{ctexart}
\usepackage[margin=2cm]{geometry}
\usepackage{listings,xcolor}
\lstset{
        language=[LaTeX]TeX,numbers=left,
        numberstyle=\tiny,
        basicstyle=\small\ttfamily,
        stringstyle=\color{purple},
        keywordstyle=\color{blue}\bfseries,
        commentstyle=\color{olive},       
        frame=shadowbox,      
        rulesepcolor=\color{red!20!green!20!blue!20}
       }
\title{使用 \LaTeX3 实现冒泡排序}
\author{ljguo1020@gmail.com}

\ExplSyntaxOn
\int_new:N \l_t_int
\NewDocumentCommand{\sort}{m}
{

\intarray_const_from_clist:Nn \l_array_int {#1}                 
\int_step_variable:nNn{\intarray_count:N \l_array_int} \i 
{
    \int_step_variable:nNn{\intarray_count:N \l_array_int-\i} \j 
    {        
        \int_compare:nT {\intarray_item:Nn \l_array_int{\j} > \intarray_item:Nn\l_array_int{\j+1}}
        {  
        \int_set_eq:NN \l_t_int \intarray_item:Nn \l_array_int{\j}   
        \intarray_gset:Nnn  \l_array_int {\j}{\intarray_item:Nn \l_array_int{\j+1}}
        \intarray_gset:Nnn  \l_array_int {\j+1}{\l_t_int}  
        }
    }
}
这些整数的由小到大的排序为:\int_step_inline:nn{\intarray_count:N \l_array_int}{\intarray_item:Nn\l_array_int {##1}~}
\cs_undefine:N \l_array_int
}
\ExplSyntaxOff 
\begin{document}
    \maketitle
学习 C 语言时学到了如下的冒泡排序法,于是用 \LaTeX3 改写了一下:
\begin{lstlisting}{language=C}
for(i=0;i<4;i++)
{
    for(j=0;j<4-i;j++)
    {
        if(a[j]>a[j+1])
        {
            t=a[j];
            a[j]=a[j+1];
            a[j+1]=t;
        }
    }
}
\end{lstlisting}
\begin{lstlisting}
\ExplSyntaxOn
\int_new:N \l_t_int
\NewDocumentCommand{\sort}{m}
{
\intarray_const_from_clist:Nn \l_array_int {#1}                 
\int_step_variable:nNn{\intarray_count:N \l_array_int} \i 
{
    \int_step_variable:nNn{\intarray_count:N \l_array_int-\i} \j 
    {        
        \int_compare:nT 
        {\intarray_item:Nn \l_array_int{\j} > \intarray_item:Nn\l_array_int{\j+1}}
        {  
        \int_set_eq:NN \l_t_int \intarray_item:Nn \l_array_int{\j}   
        \intarray_gset:Nnn  \l_array_int {\j}{\intarray_item:Nn \l_array_int{\j+1}}
        \intarray_gset:Nnn  \l_array_int {\j+1}{\l_t_int}  
        }
    }
}
\int_step_inline:nn{\intarray_count:N\l_array_int}{\intarray_item:Nn\l_array_int {##1}~}
\cs_undefine:N \l_array_int
}
\ExplSyntaxOff
\end{lstlisting}
\begin{lstlisting}
\sort{15,4,12,5,45,5,6,15}\par 
\sort{12,5,4,5,6,78,100,105}\par 
\sort{42,0,-3,5,25,68}
\end{lstlisting}

\sort{15,4,12,5,45,5,6,15}\par 
\sort{12,5,4,5,6,78,100,105}\par 
\sort{42,0,-3,5,25,68}
\end{document}