\documentclass{mycls}
\title{积分入门}
\author{芒果不盲}
\vision{10.20版}
\slogan{ 每一个积分都蕴藏着一个美妙的故事}

            \newcommand{\solutionname}{解}
            \newenvironment{solution}{\par\noindent\textbf{\color{red}\solutionname} \kaishu}{\par}

 
    

 

\begin{document}



\maketitle
\begin{center}
 \Huge   \bfseries \heiti 前言
\end{center}
\tableofcontents
\newpage
{\xingkai \zhlipsum[1]}
sedfds当时法国的后果 
\chapter{不定积分}
\section{第一类换元法}
\section{第二类换元法}
\subsection{三角代换}
\subsection{根代换}
\subsection{双曲代换}
\section{欧拉替换}
\section{分部积分}
\subsection{分部积分法}
\subsection{推广(表格法)}
\section{有理函数积分}
\section{三角高次}
\subsection{降幂法}
\subsection{De Moivre}
\section{三角有理式}
\subsection{升幂法}
\subsection{\textcolor{purple}{组合积分法}}
\chapter{定积分}
\section{定积分定义}
\section{牛顿莱布尼兹公式}
\section{换技}
\chapter{定积分}
\begin{solution}
    \circled{\small 变}
   
\end{solution}
\begin{align*}
   \textcolor{qianzi}{\int_{0}^{+\infty} e^{-pt}\sin {\omega}t\,\mathrm{d}t} &=-\dfrac{1}{\omega}\int_{0}^{+\infty}e^{-pt}\,\mathrm{d}(\cos{\omega t})\\
   &=-\dfrac{1}{\omega}\Biggl[e^{-pt}\cos{\omega t}\Biggl|_{0}^{+\infty}-\int_{0}^{+\infty}\cos{\omega t}\,\mathrm{d}(e^{-pt})\Biggr]\\
   &=-\dfrac{1}{\omega}\Biggl[-1+p\int_{0}^{+\infty }e^{-pt}\cos{\omega t}\mathrm{d}t\Biggr] \\
   &=-\dfrac{1}{\omega}\Biggl[-1+\dfrac{p}{\omega}\int_{0}^{+\infty}e^{-pt}\,\mathrm{d}(\sin{\omega t})\Biggr]\\
   &=-\dfrac{1}{\omega}\Biggl[-1+\dfrac{p}{\omega}\Bigl(e^{-pt}\sin{\omega t}\Biggl|_{0}^{+\infty}-\int_{0}^{+\infty}\sin{\omega t}\,\mathrm{d}(e^{-pt})\Bigr)\Biggr]\\
   &=-\dfrac{1}{\omega}\Biggl[-1+\dfrac{p}{\omega}\Bigl(0+p \textcolor{qianzi}{\int_{0}^{+\infty}e^{-pt}\sin{\omega t}\,\mathrm{d}t}\Bigr)\Biggr]
\end{align*}
自然得到:
\begin{align*}
    \int_{0}^{+\infty}e^{-pt}\sin{\omega t}\,\mathrm{d}t=\dfrac{\omega}{p^{2}+\omega ^{2}}  
\end{align*}


\newpage
\begin{tikzpicture}
\draw[-latex] (-0.3,0)--(3.1,0);
\draw[-latex] (0,-0.3)--(0,1.7);
\coordinate (a) at (0.2,0);
\coordinate (b) at (2.8,0);
\draw[fill=hailan,draw=red](0.2,0)-- (0.2,0.8) .. controls (0.7,1.8)and(1.3,0.1)  .. (2.8,1.6)--(2.8,0)--cycle;

\draw[] (a) node[below]{$a$};
\draw[](b)  node [below]{$b$};
\draw[] (1.6,1.5) node[scale=0.3]{$ \int_{a}^{b}f(x)\,\mathrm{d}x=F(b)-F(a) $  };
\draw[](3.1,0) node[below,right] {$x$};
\draw[](0,1.7)  node[above]{$y$};
\end{tikzpicture}
\end{document} 
